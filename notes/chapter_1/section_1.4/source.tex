\section{Consequences of Completeness}
    \begin{theorem}[Nested Interval Property]
        For each $n \in \textbf{N}$, assume we are given a closed interval $I_n = [a_n, b_n] = \{x \in \textbf{R}: a_n \leq x \leq b_n\}$. Assume also that each $I_n$ contains $I_{n+1}$. Then, the resulting nested sequence of closed intervals
        \begin{equation*}
            I_1 \supseteq I_2 \supseteq I_3 \supseteq I_4 \supseteq \dots
        \end{equation*}
        has a nonempty intersection; that is, $\bigcap_{n=1}^{\infty} I_n \neq \emptyset$.
    \end{theorem}
    \begin{proof}
        In order to show $\bigcap_{n=1}^{\infty} I_n$ is not empty, we are going to use the Axiom of Completeness to produce a single real number $x$ satisfying $x \in I_n$ for every $n \in \textbf{N}$. Consider the set
        \begin{equation*}
            A = \{a_n : n \in \textbf{N}\}
        \end{equation*}
        of left-hand endpoints of the intervals. Since the intervals are nested, every $b_n$ is an upper bound for $A$. let $x = \text{sup } A$. Consider a particular $I_n = [a_n, b_n]$. Since $x$ is an upper bound for $A$, $a_n \leq x$. Since $x$ is the least upper bound and each $b_n$ are upper bounds, $x \leq b_n$. So $a_n \leq x \leq b_n$ for any $n$. So $x \in I_n$ for any $n \in \textbf{N}$. Hence, $x \in \bigcap_{n=1}^{\infty} I_n \neq \emptyset$.
    \end{proof}
    \subsection*{The Density of Q in R}
        \begin{theorem}[Archimedean Property]
            (i) Given any number $x \in \textbf{R}$, there exists an $n \in \textbf{N}$ satisfying $n > x$. 
            \newline \indent (ii) Given any real number $y > 0$, there exists an $n \in \textbf{N}$ satisfying $1/n < y$.
        \end{theorem}
        \begin{proof}
            Part (i) states that \textbf{N} is not bounded above. Assume, for contradiction, that \textbf{N} is bounded above. By AoC, \textbf{N} has a least upper bound. Let $\alpha = \text{sup } N$. $\alpha - 1$ is not an upper bound, so there is an $n \in \textbf{N}$, such that $\alpha - 1 < n$, which is the same as saying $\alpha < n + 1$. $n + 1 \in \textbf{N}$, we have a contradiction to the fact $\alpha$ is an upper bound. 
            \newline \indent Part (ii) follows from (i) by letting $x = 1 / y$.
        \end{proof}
        \begin{theorem}[Density of Q in R]
            For every two real numbers a and b with $a < b$, there exists a rational number r satisfying $a < r < b$.
        \end{theorem}
        \begin{proof}
            To simplify matters, let's assume $0 \leq a < b$. A rational number is a quotient of integers, so we must product $m$, $n \in \textbf{N}$ so that 
            \begin{equation*}
                a < \frac{m}{n} < b
            \end{equation*}
            First, we must choose a large enough $n$ so that an increment of size $1/n$ is small enough so it doesn't step over the interval (a, b). Basically, we need an $n \in \textbf{N}$ such that
            \begin{equation*}
                \frac{1}{n} < b - a
            \end{equation*}
            By the first inequality, we can get $na < m < nb$. With $n$ chosen, we need to choose an m to be the smallest natural number greater than $na$. So,
            \begin{equation*}
                m - 1 \leq na < m
            \end{equation*}
            which yields $a < m/n$. And $a < b - 1/n$ from the second inequality. So
            \begin{equation*}
                m \leq na + 1 < n(b - \frac{1}{n}) + 1 = nb
            \end{equation*}
            Because $m < nb$ so $m/n < b$. Now we have $a < m/n < b$.
        \end{proof}
        \textbf{Collary} \textit{Given any two real numbers $a < b$, there exists an irrational number t satisfying $a < t < b$}
        \subsection*{The Existence of Square Roots}
        \begin{theorem}
            There exists a real numbers $\alpha \in \textbf{R}$ satisfying $\alpha ^ 2 = 2$.
        \end{theorem}
        \begin{proof}
            Consider the set
            \begin{equation*}
                T = \{t \in \textbf{R}: t^2 < 2\}
            \end{equation*}
            and set $\alpha = \text{sup } T$. If $\alpha ^ 2 < 2$. NEED TO FINISH THIS PROOF.
        \end{proof}
    \subsection*{Countable and Uncountable Sets}
        \textbf{Cardinality}
            \newline
            \textit{Cardinality} refers to the size of a set. The cardinalities of finite sets can be compared by attaching a natural number to each set. By using comparisons rather than just length, this idea extends to infinite sets.
            \newline  \newline
            \textbf{Definition} A function $f: A \rightarrow B$ is one-to-one (1-1) if $a_1 \neq a_2$ in $A$ implies that $f(a_1) \neq f(a_2)$ in $B$. The function $f$ is \textit{onto} if given any $b \in B$, it is possible to find the element $a \in A$ such that $f(a) = b$.
            \textbf{Definition} Two sets $A$ and $B$ havethe same cardinality if there exists $f: A \rightarrow B$ that is 1-1 and onto. In this case, we write $A \sim B$.
            \newline
        \textbf{Countable Sets}
            \newline
            \textbf{Definition} A set $A$ is \textit{countable} if $N \sim A$. AN infinite set that is countable is called an \textit{uncountable} set.
            \begin{theorem}
                (i) The set \textbf{Q} is countable
                \newline
                (ii) The set \textbf{R} is uncountable
            \end{theorem}
            \begin{proof}
                (i) For each $n \in \textbf{N}$, let
                \begin{equation*}
                    A_n = \{\pm \frac{p}{q}: \text{where } p, q \in \textbf{N} \text{ are in lowest terms with } p + q = n\}
                \end{equation*}
                so
                \begin{equation*}
                    A_1 = \{\frac{0}{1}\},\indent
                    A_2 = \{\frac{1}{1}, \frac{-1}{1}\}, \indent
                    A_3 = \{\frac{1}{2}, \frac{-1}{2}, \frac{2}{1}, \frac{-2}{1}\}
                \end{equation*}
                Our one to one correspondence from \textbf{N} to \textbf{Q} is by listing the elements from $\bigcup_{n=1}^{\infty} A_n$. So, $f(n) = (\bigcup_{n=1}^{\infty} A_n)[n]$. For any fraction, like 22/7, it will be in $\bigcup_{n=1}^{\infty} A_n$ exactly once ($22/7 \in A_29$). This makes $\bigcup_{n=1}^{\infty}\bigcup_{m=1}^{\infty} A_n \cap A_m = \emptyset$. So, $\textbf{N} \sim \textbf{Q}$ and \textbf{Q} is countable.
                \newline \indent
                (ii) Proof by contradiction. Assume there exists a 1-1 from \textbf{N} to \textbf{R}. If we let $x_n = f(n)$ for each $n \in \textbf{N}$, we can write
                \begin{equation*}
                    \textbf{R} = \{x_1, x_2, x_3, \dots \}
                \end{equation*}
                Let $I_1$ be a closed interval that does not contain $x_1$. Then create infinite intervals based on the following rules. Given an $I_n$, construct $I_{n+1}$ to satisfy
                \begin{equation*}
                    \text{(i) } I_{n+1} \subseteq I_n \text{ and}
                \end{equation*}
                \begin{equation*}
                    \text{(ii) } x_{n+1} \notin I_{n+1}.
                \end{equation*}
                Given $I_n$, it is clear that $I_{n+1}$ exists since $I_n$ certainly contains two smaller disjoint closed intervals and $x_{n+1}$ can only be in one of them. Since $x_{n_0} \notin I_{n_0}$,
                \begin{equation*}
                    x_{n_0} \notin \bigcap_{n=1}^{\infty} I_n
                \end{equation*}
                This is true for every natural number $n_0$, and hence every real number $x_{n_0}$, so
                \begin{equation*}
                    \bigcap_{n=1}^{\infty} I_n = \emptyset
                \end{equation*}
                which contradicts the Nested Interval Property, which asserts that $\bigcap_{n=1}^{\infty} I_n \neq \emptyset$. Due to this contradiction, \textbf{R} cannot be countable, and is uncountable.
            \end{proof}
            Since $\textbf{R} = \textbf{Q} \cup \textbf{I}$, where \textbf{I} is all irrational numbers, \textbf{I} cannot be countable because otherwise \textbf{R} would be.
            \begin{theorem}
                If $A \subseteq B$ and B is countable, then A is either countable, finite, or empty.
            \end{theorem}
            \begin{theorem}
                (i) If $A_1$, $A_2$, \dots $A_m$ are each countable sets, then the union $\bigcup_{n=1}^{m} A_n$ is countable.
                \newline
                (ii) If $A_n$ is a countable set for each $n \in \textbf{N}$, then $\bigcup_{n=1}^{\infty} A_n$ is countable.
            \end{theorem}